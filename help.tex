\documentclass[12pt, letterpaper]{article}

\usepackage{tabularx}
\usepackage{hyperref}
%\usepackage[utf8]{inputenc}
\usepackage[T1]{fontenc}
\usepackage{array}
\usepackage{listings}

\lstset{
  numbers=left,
  numberstyle=\small,
  numbersep=8pt,
  language=Lisp,
  stringstyle=\ttfamily\small,
  basicstyle=\ttfamily\footnotesize,
  showstringspaces=false,
  frame= single,
  framexleftmargin= 15pt,
  breaklines
}


\hypersetup{
    colorlinks,
    citecolor=blue,
    filecolor=blue,
    linkcolor=blue,
    urlcolor=blue
} 

% set the depth of table of contents
\setcounter{tocdepth}{3}
\setcounter{secnumdepth}{3}

% column types (sets ragged alignment and removes justified spacing)
\newcolumntype{L}[1]{>{\raggedright\let\newline\\\arraybackslash\hspace{0pt}}m{#1}}
\newcolumntype{C}[1]{>{\centering\let\newline\\\arraybackslash\hspace{0pt}}m{#1}}
\newcolumntype{R}[1]{>{\raggedleft\let\newline\\\arraybackslash\hspace{0pt}}m{#1}}

% use this to write code in text
\def\code#1{\texttt{#1}}

\title{sp00ky's Help document}
\author{sp00kyb00g13}
 
\begin{document}

\begin{titlepage}
\maketitle
\end{titlepage}

\tableofcontents
\clearpage

\section{GNU/Linux}
\subsection{Documentation}
\begin{tabular} {|L{0.2\textwidth}|L{0.8\textwidth}|}
  \hline
  \textbf{Function Name} &
  \textbf{Description} \\\hline\hline
  man hier & Information on filesystem \\\hline

\end{tabular}

\subsection{Command Line}
\begin{tabular} {|L{0.2\textwidth}|L{0.4\textwidth}|L{0.4\textwidth}|}
  \hline
  \textbf{Function Name} &
  \textbf{Description} &
  \textbf{Example} \\\hline\hline
  detex & strips tex commands from file & detex <filename> \\\hline
  diction & Prints misused words & detex <filename> | diction -bs \\\hline
  type & prints info about a command & type cd \\\hline

\end{tabular}

\section{Emacs}
General tips on emacs, probably will contain a lot of forum posts.
\subsection{Commands or How to use them}
\subsubsection{Elisp Navigation}
\begin{tabular} {|L{0.2\textwidth}|L{0.4\textwidth}|L{0.4\textwidth}|}
  \hline
  \textbf{Function Name} &
  \textbf{Description} &
  \textbf{Example} \\\hline\hline
  find-function & Finds elisp fn tag & interactive \\\hline
\end{tabular}

\subsubsection{Align-regexp}
This \href{https://emacs.stackexchange.com/questions/2644/understanding-of-emacs-align-regexp}{post} gives a good explanation of using \code{align-regexp}

\subsection{AUCTeX}
% enter stuff about aspell-en on arch for flyspell
\code{C-c RET} - TeX-insert-macro to insert macros (/href etc)
\code{C-c C-e} - LaTeX-environment insert stuff

\subsection{Emacs Startup Time}
Read the comments in \href{https://github.com/abo-abo/profile-dotemacs/blob/master/profile-dotemacs.el}{this} file to see how to evaluate startup time. \\
\subsubsection{Daemon}
Emacs can be started as a daemon to boost its startup time. Lots of info can be found \href{https://www.emacswiki.org/emacs/EmacsAsDaemon#toc12}{here}.
\begin{lstlisting}{caption= Example Emacs Daemon Usage}
emacs --daemon
emacsclient -nw
\end{lstlisting}
To kill the daemon simply use \code{M-x kill-emacs}. \\
To run multiple emacs client observe the following.
\begin{lstlisting}
emacs --daemon=<my-server-name> 
emacsclient -nw -s <my-server-name>
\end{lstlisting}

\subsection{Color-Themes}
\begin{tabular} {|L{0.2\textwidth}|L{0.4\textwidth}|L{0.4\textwidth}|}
  \hline
  \textbf{Function Name} &
  \textbf{Description} &
  \textbf{Example} \\\hline\hline
  what-cursor-position & With a prefix arg (\code{C-u}) it will describe the face at the cursor & \code{C-u M-x what-cursor-position} \\\hline
\end{tabular}

\section{ELisp Help}
Most of the information here is taken from \href{https://www.gnu.org/software/emacs/manual/html_node/elisp/index.html}{The Emacs manual}
\subsection{Useful Functions}

\subsubsection{Misc}
Functions that are usefull but don't deserve their own subsubsection. \\

\begin{tabular} {|L{0.2\textwidth}|L{0.4\textwidth}|L{0.4\textwidth}|}
  %{|p{0.2\textwidth}|p{0.4\textwidth}|p{0.4\textwidth}|}
  \hline
  \textbf{Function Name} &
  \textbf{Description} &
  \textbf{Example} \\\hline\hline
  \href{https://www.gnu.org/software/emacs/manual/html_node/eintr/fwd_002dpara-let.html#fwd_002dpara-let}{let*} & like \code{let} but emacs set each variable in sequence, variables at the end can make use of variables at the start & n/a \\\hline
\end{tabular}

\subsubsection{Minibuffer}

\begin{tabular} {|L{0.2\textwidth}|L{0.4\textwidth}|L{0.4\textwidth}|}
  %{|p{0.2\textwidth}|p{0.4\textwidth}|p{0.4\textwidth}|}
  \hline
  \textbf{Function Name} &
  \textbf{Description} &
  \textbf{Example} \\\hline\hline
  read-from-minibuffer & Prompts for info in the minibuffer & \code{read-from-minibuffer 'Enter Your Name'} \\\hline
\end{tabular}

\subsubsection{Buffers}
\begin{tabular} {|L{0.2\textwidth}|L{0.4\textwidth}|L{0.4\textwidth}|}
  %{|p{0.2\textwidth}|p{0.4\textwidth}|p{0.4\textwidth}|}
  \hline
  \textbf{Function Name} &
  \textbf{Description} &
  \textbf{Example} \\\hline\hline
  buffer-name & Gets current buffer name & n/a  \\\hline
  other-buffer & Gets previous buffer name & n/a  \\\hline
  switch-to-buffer & Changes emacs focused buffer (the display buffer) & \code{(switch-to-buffer (other-buffer))} \\\hline
  set-buffer & Changes the attention of the program to a different buffer (doesnt change the displayed buffer) & \code{(set-buffer (other-buffer))} \\\hline
  beggining-of-buffer & goto beginning of buffer and leave mark on previous position & n/a \\\hline
  save-excursion & performs body then returns to point & i\code{save-excursion (let *...)} \\\hline
\end{tabular}

\subsubsection{Windows}
Neotree uses something called a dedicated buffer
\begin{tabular} {|L{0.2\textwidth}|L{0.4\textwidth}|L{0.4\textwidth}|}
  %{|p{0.2\textwidth}|p{0.4\textwidth}|p{0.4\textwidth}|}
  \hline
  \textbf{Function Name} &
  \textbf{Description} &
  \textbf{Example} \\\hline\hline
  split-window & use \code{C-h f} & \code{split-window nil 10 'right} \\\hline
\end{tabular}

\subsubsection{Lists}
\begin{tabular} {|L{0.2\textwidth}|L{0.4\textwidth}|L{0.4\textwidth}|}
  \hline
  \textbf{Function Name} &
  \textbf{Description} &
  \textbf{Example} \\\hline\hline
  \href{https://www.gnu.org/software/emacs/manual/html_node/eintr/car-cdr-_0026-cons.html}{cdr} & Manipulate list & n/a \\\hline
  \href{https://www.gnu.org/software/emacs/manual/html_node/elisp/Mapping-Functions.html}{mapcar} & Apply function to sequence, i.e apply function to every element in sequence & \code{(mapcar '1+ [1 2 3])} \\\hline
\end{tabular}

\subsubsection{Markers}
\label{sssec:markers}
Mark is a position in the buffer.

\begin{tabular} {|L{0.2\textwidth}|L{0.4\textwidth}|L{0.4\textwidth}|}
  \hline
  \textbf{Function Name} &
  \textbf{Description} &
  \textbf{Example} \\\hline\hline
  \hline
  mark-whole-buffer & n/a & n/a \\\hline
  save-excursion & performs body then returns to point & i\code{save-excursion (let *...)} \\\hline
  \hline
\end{tabular}

\subsubsection{Points}
Point is where the cursor is. Part of the buffer between point and a \hyperref[sssec:markers]{mark} is called \code{region}

\begin{tabular} {|L{0.2\textwidth}|L{0.4\textwidth}|L{0.4\textwidth}|}
  \hline
  \textbf{Function Name} &
  \textbf{Description} &
  \textbf{Example} \\\hline\hline
  \hline
  save-excursion & Saves the location of point, executes the body of the function and restores point (also restores the buffer). &
  \code{(save-excursion
    body...)}
  or 
 \code{(let varlist
  (save-excursion
           body...))}\\
  \hline
\end{tabular}


\subsection{Writing Elisp}
\subsubsection{Functions}
Here are 5 important points when writing a function:

\begin{enumerate}
\item The name of the symbol to which the function definition should be attached.
\item A list of the arguments that will be passed to the function. If no arguments will be passed to the function, this is an empty list, (). 
\item Documentation describing the function. (Technically optional, but strongly recommended.)
\item Optionally, an expression to make the function interactive so you can use it by typing M-x and then the name of the function; or by typing an appropriate key or keychord.
\item The code that instructs the computer what to do: the body of the function definition.
\end{enumerate}

\begin{lstlisting}[caption=Function skeleton]
  function-name (arguments...)
  ``optional-documentation...''
  (interactive argument-passing-info)     ; optional
  body...)
\end{lstlisting}

\subsubsection{Example}
The following is an example of a function and its usage, to use it evaluate the function using \code{C-x C-e} then evaluate the following expression:

\begin{lstlisting}[caption= Multiply by seven example]
  (defun multiply-by-seven (num) ; interactive
  "Multiply numbers  by 7"
  (interactive "p")
  (message "The result is \%d" (* 7 num)))

  (multiply-by-seven 3)
\end{lstlisting}

The following is an example of an interactive function, to use it do \code{C-u <number> M-x multiply-by-seven} after you have evaluated the function using \code{C-x C-e}
\begin{lstlisting}[caption= Interactive multiply by seven example]
  (defun multiply-by-seven (number)       ; Interactive version.
  ``Multiply NUMBER by seven.''
  (interactive ``p'')
         (message ``The result is %d'' (* 7 number)))
\end{lstlisting}

\subsubsection{Scope of Variables}
Using \code{let} let sets the scope of the variable to that let scope (inside \code{let}'s brackets). Here is an example of how to use let:

\begin{lstlisting}[caption= Let general usage]
  (let ((variable value)
  (variable value)
  ...)
         body...)))
\end{lstlisting}

\subsubsection{If}
Everything except nil is true. If statement works as expected, heres an example:

\begin{lstlisting}[caption= If statement example with int]
  (if (> 5 4)                                  ; if-part
           (message ``5 is greater than 4!'')) ; then-part
\end{lstlisting}

Here is an example using a string:

\begin{lstlisting}[caption= If statement example with string]
  (defun type-of-animal (characteristic)
  ``Print message in echo area depending on CHARACTERISTIC.
  If the CHARACTERISTIC is the string \"fierce\",
  then warn of a tiger.''
  (if (equal characteristic ``fierce'')
             (message ``It is a tiger!'')))
\end{lstlisting}

Here is an example of an if-else:

\begin{lstlisting}[caption= If-else statement with int]
  (if (> 4 5)                                    ; if-part
  (message ``4 falsely greater than 5!'')        ; then-part
         (message ``4 is not greater than 5!'')) ; else-part
\end{lstlisting}

\subsubsection{Nil}
Nil has two meanings, false and empty list, referred to as \code{nil} or \code{()}. These mean the same thing.

\end{document}
